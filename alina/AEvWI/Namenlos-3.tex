%5.Uebungsblatt Algorithmische Eigenschaften von Wahlsystemen I WS 10/11
\documentclass[12pt]{article}

\usepackage[ngerman]{babel}
\usepackage[T1]{fontenc}
\usepackage[ansinew]{inputenc}
\usepackage{times}

\usepackage{color}
%\usepackage{psfig}
\usepackage{epsfig}  
\usepackage{amsfonts}
\usepackage{amsmath}
\usepackage{amssymb}
\usepackage{eepic,epic}
\usepackage{multirow}
\usepackage{fancyhdr}


\setlength{\oddsidemargin}{0.25in}
\setlength{\evensidemargin}{\oddsidemargin}
\setlength{\textwidth}{6in}
\setlength{\textheight}{8.6in}
\setlength{\topmargin}{-0.0in}


\newcommand\bc{\begin{center}}
\newcommand\ec{\end{center}}

\newcommand\beq{\begin{eqnarray*}}
\newcommand\eeq{\end{eqnarray*}}

\newcommand\be{\begin{enumerate}}
\newcommand\ee{\end{enumerate}}

\renewcommand{\labelenumi}{(\alph{enumi})}

\newcommand{\N}{\mathbb{N}}
\newcommand{\Z}{\mathbb{Z}}

\newcommand{\bigoh}{\mathcal{O}}

% COMPLEXITY CLASSES
\newcommand{\DTIME}[1]{{\mbox{\rm DTIME}}(#1)}
\newcommand{\DSPACE}[1]{{\mbox{\rm DSPACE}}(#1)}
\newcommand{\NTIME}[1]{{\mbox{\rm NTIME}}(#1)}
\newcommand{\NSPACE}[1]{{\mbox{\rm NSPACE}}(#1)}
\newcommand{\ATIME}[1]{{\mbox{\rm ATIME}}(#1)}
\newcommand{\ASPACE}[1]{{\mbox{\rm ASPACE}}(#1)}

\newcommand{\realtime}{\mbox{\rm REALTIME}}
\newcommand{\lintime}{\mbox{\rm LINTIME}}
\newcommand{\polylogspace}{\mbox{\rm POLYLOGSPACE}}
\newcommand{\alogtime}{\mbox{\rm ALOGTIME}}
\newcommand{\linspace}{\mbox{\rm LINSPACE}}
\newcommand{\nlinspace}{\mbox{\rm NLINSPACE}}
\newcommand{\np}{\mbox{\rm NP}}

\newcommand{\red}{\leq_{m}^{p}}
\newcommand{\vs}{\mathcal{E}}
\newcommand{\wahl}{(C,V)}
\newcommand{\wahlind}[1]{(C_{#1},V_{#1})}
\newcommand{\subwahlv}[1]{(C,V_{#1})}
\newcommand{\subwahl}[1]{(C_{#1},V)}
\newcommand{\xSet}[2]{\{#1_1,#1_2,\ldots,#1_{#2}\}}
\newcommand{\xList}[2]{(#1_1,#1_2,\ldots,#1_{#2})}

\newcommand{\scoresub}[2]{\mathit{score}_{#1}(#2)}
\newcommand{\scoreof}[1]{\mathit{score}(#1)}

\newcommand{\bAnt}{Begr\"unden Sie Ihre Antwort.}
%\pagestyle{empty}
\begin{document}
Prof. J.~Rothe
\hfill D\"usseldorf, 11.11.2010
\vspace{0.5cm}

\begin{center}
{\Large \"Ubung zur Vorlesung im WS 2010/2011} \\
{\LARGE \bf Algorithmische Eigenschaften von Wahlsystemen I} \\
Blatt 5, Abgabe am 18. November 2010\\
\end{center}
\vspace{0.5cm}

%%%%%%%%%%%%%%%%%%%%%%%%%%%%%%%
%%%%%%%%   Aufgabe 1   %%%%%%%%
%%%%%%%%%%%%%%%%%%%%%%%%%%%%%%%
\noindent{\bf Aufgabe 1 (Condorcet-Verlierer):} 
Der \emph{Condorcet-Verlierer} einer Wahl ist derjenige Kandidat, der im paarweisen Vergleich 
von allen anderen Kandidaten in mehr als der H\"alfte der Stimmen geschlagen wird.
\be
	\item Zeigen Sie, dass ein Condorcet-Verlierer ein Plurality-Gewinner sein kann. \bAnt 
	\item Zeigen Sie, dass ein Borda-Gewinner nie ein Condorcet-Verlierer ist. \bAnt
\ee
\textbf{Hinweis zu (b):} Wir definieren $N(c,d)=\|\{v\in V \mid \mbox{$c$ steht in $v$ vor $d$}\}\|$.
Zeigen Sie zun\"achst, dass f\"ur eine Borda-Wahl $\wahl$ folgendes gilt:
\begin{equation}
	\mathit{Bscore}(c) = \sum\limits_{d\in C-\{c\}}N(c,d), \label{eq:eq1}
\end{equation}
wobei $\mathit{Bscore}(c)$ der Borda-Score von Kandidat $c$ ist. Beweisen Sie daraufhin mithilfe von
(\ref{eq:eq1}) die Aussage in (b) durch Widerspruch.

\vspace{0.5cm}

%%%%%%%%%%%%%%%%%%%%%%%%%%%%%%%
%%%%%%%%   Aufgabe 2   %%%%%%%%
%%%%%%%%%%%%%%%%%%%%%%%%%%%%%%%
\noindent{\bf Aufgabe 2 (Theorem von Muller und Satterthwaite):}
Warum wird die Resolut-Eigenschaft f\"ur die G\"ultigkeit des Theorems von Muller und
Satterthwaite ben\"otigt? \bAnt

\vspace{0.5cm}

\noindent{\bf Aufgabe 3 (Unabh\"angigkeit von irrelevanten Alternativen):} Ein Wahlsystem $\vs$
ist \emph{unabh\"angig von irrelevanten Alternativen (erf\"ullt die IIA-Eigenschaft)}, wenn 
f\"ur jede $\vs$-Wahl gilt: Ist $c$ im Gesamtergebnis der Wahl vor $d$ platziert, so \"andert 
sich an dieser Reihenfolge nichts, auch wenn die Stimmen in $V$ derart ver\"andert werden, 
dass lediglich die Reihenfolge von $c$ und $d$ erhalten bleibt. 

Es sei $(C,V)$ die 
Wahl von Blatt 2: Die Kandidatenmenge sei $C=\{a,b,c,d,e\}$, die W\"ahlermenge $V$ enthalte
sechs W\"ahler mit den folgenden Pr\"aferenzen:
\[
\begin{array}{rc@{\ \ }ccccc}
	v_1: 	& d & c & a & e & b	\\
	v_2:  & d & c & b & a & e 	\\
	v_3:  & d & b & e & a & c 	\\
	v_4:	& e & c & b & a & d 	\\
	v_5: 	& b & c & a & d & e	\\
	v_6: 	& a & c & b & d & e	\\
\end{array}
\]

\be
	\item Stellen Sie f\"ur Borda, Plurality Voting und Veto das Ergebnis der 
	Wahl $\wahl$ als Pr\"aferenz dar. (Gleichst\"ande werden mit \glqq =\grqq\, gekennzeichnet.)
	\item Zeigen Sie anhand dieser Wahl, dass Borda, Plurality Voting und Veto nicht
	unabh\"angig von irrelevanten Alternativen sind. \bAnt
	\item Angenommen, wir definieren das strenge Monotonie-Kriterium analog zu der IIA-Eigenschaft
	f\"ur den paarweisen Vergleich von Kandidaten:
	
	Ein Wahlsystem $\vs$ sei streng monoton, wenn f\"ur jede $\vs$-Wahl $\wahl$ gilt: Steht im Gesamtergebnis
	der Wahl $c$ vor $d$, so bleibt diese Reihenfolge erhalten, selbst wenn in $V$ die Stimmen derart 
	ver\"andert werden, dass lediglich gew\"ahrleistet ist, dass Kandidaten, die vor der \"Anderung hinter 
	$c$ positioniert waren, auch nach der \"Anderung hinter $c$ positioniert sind.

	Worin besteht der Unterschied zwischen der IIA-Eigenschaft und der so definierten strengen 
	Monotonie? Was k\"onnen Sie daraus f\"ur den Zusammenhang zwischen diesen beiden Eigenschaften folgern? 
	\bAnt
\ee
\end{document}

