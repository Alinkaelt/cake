%Loesungen 6. Uebungsblatt Algorithmische Eigenschaften von Wahlsystemen I WS 10/11
\documentclass[a4paper,12pt,titlepage,oneside]{article}

%\usepackage[lmargin=2.00cm,rmargin=2.00cm,tmargin=3cm,bmargin= 3cm]{geometry}
\usepackage[ngerman]{babel}
\usepackage[T1]{fontenc}
\usepackage[ansinew]{inputenc}
\usepackage{times}

\usepackage{color}
\usepackage{picture}
%\usepackage{psfig}
%\usepackage{pst-all}
\usepackage{epsfig}  
\usepackage{eqlist}
\usepackage{amsfonts}
\usepackage{amsmath}
\usepackage{amssymb}
%\usepackage{eepic,epic}
\usepackage{tabularx,multirow}
\usepackage{fancyhdr}
\usepackage{graphics}
%\usepackage[all,dvips,arc,curve,frame,graph]{xy}

\setlength{\oddsidemargin}{0.25in}
\setlength{\evensidemargin}{\oddsidemargin}
\setlength{\textwidth}{6in}
\setlength{\textheight}{8.6in}
\setlength{\topmargin}{-0.0in}


\newcommand\bc{\begin{center}}
\newcommand\ec{\end{center}}

\newcommand\beq{\begin{eqnarray*}}
\newcommand\eeq{\end{eqnarray*}}

\newcommand\be{\begin{enumerate}}
\newcommand\ee{\end{enumerate}}

\renewcommand{\labelenumi}{(\alph{enumi})}

\newcommand{\N}{\mathbb{N}}
\newcommand{\Z}{\mathbb{Z}}

\newcommand{\bigoh}{\mathcal{O}}

% COMPLEXITY CLASSES
\newcommand{\DTIME}[1]{{\mbox{\rm DTIME}}(#1)}
\newcommand{\DSPACE}[1]{{\mbox{\rm DSPACE}}(#1)}
\newcommand{\NTIME}[1]{{\mbox{\rm NTIME}}(#1)}
\newcommand{\NSPACE}[1]{{\mbox{\rm NSPACE}}(#1)}
\newcommand{\ATIME}[1]{{\mbox{\rm ATIME}}(#1)}
\newcommand{\ASPACE}[1]{{\mbox{\rm ASPACE}}(#1)}

\newcommand{\realtime}{\mbox{\rm REALTIME}}
\newcommand{\lintime}{\mbox{\rm LINTIME}}
\newcommand{\polylogspace}{\mbox{\rm POLYLOGSPACE}}
\newcommand{\alogtime}{\mbox{\rm ALOGTIME}}
\newcommand{\linspace}{\mbox{\rm LINSPACE}}
\newcommand{\nlinspace}{\mbox{\rm NLINSPACE}}
\newcommand{\NP}{\mbox{\rm NP}}

\newcommand{\reduc}{\leq_{m}^{p}}
\newcommand{\vs}{\mathcal{E}}
\newcommand{\wahl}{(C,V)}
\newcommand{\wahlind}[1]{(C_{#1},V_{#1})}
\newcommand{\subwahlv}[1]{(C,V_{#1})}
\newcommand{\subwahl}[1]{(C_{#1},V)}
\newcommand{\xSet}[2]{\{#1_1,#1_2,\ldots,#1_{#2}\}}
\newcommand{\xList}[2]{(#1_1,#1_2,\ldots,#1_{#2})}
\newcommand{\xSetfour}[1]{\{#1_1,#1_2,#1_3,#1_4\}}
\newcommand{\xListfour}[1]{(#1_1,#1_2,#1_3,#1_4)}
\newcommand{\coll}{\mathcal{S}}

\newcommand{\scoresub}[2]{\mathit{score}_{#1}(#2)}
\newcommand{\scoreof}[1]{\mathit{score}(#1)}

\newcommand{\bAnt}{Begr\"unden Sie Ihre Antwort.}
\newcommand{\seq}{\subseteq}
\newcommand{\setrestr}[2]{\{#1 \mid #2\}}

\newcommand{\xcovthree}{{\sc Exact Cover By Three Sets}}
\newcommand{\xcthree}{{\sc X3C}}
\newcommand{\setcover}{{\sc Set Cover}}
\newcommand{\setc}{{\sc SC}}
\newcommand{\hittingset}{{\sc Hitting Set}}
\newcommand{\hs}{{\sc HS}}
\newcommand{\subsetsum}{{\sc Subset Sum}}
\newcommand{\subsets}{{\sc SS}}
\newcommand{\partition}{{\sc Partition}}

\newcommand{\xcovinst}{(B,\coll)}
\newcommand{\scovinst}{(B,\coll,k)}
\newcommand{\hsinst}{(B,\coll,k)}
\newcommand{\partinst}{(A,s)}
\newcommand{\ssinst}{(A,s,k)}


\newcommand{\ccwmani}[1]{{\sc #1-Constructive Coalitional Weighted Manipulation}}
\newcommand{\dcwmani}[1]{{\sc #1-Destructive Coalitional Weighted Manipulation}}
\newcommand{\ccmani}[1]{{\sc #1-Constructive Coalitional Manipulation}}
\newcommand{\dcmani}[1]{{\sc #1-Destructive Coalitional Manipulation}}
\newcommand{\cwmani}[1]{{\sc #1-Constructive Weighted Manipulation}}
\newcommand{\dwmani}[1]{{\sc #1-Destructive Weighted Manipulation}}
\newcommand{\cmani}[1]{{\sc #1-Constructive Manipulation}}
\newcommand{\dmani}[1]{{\sc #1-Destructive Manipulation}}

\newcommand{\ccwmanishort}[1]{{\sc #1-CCWM}}
\newcommand{\dcwmanishort}[1]{{\sc #1-DCWM}}
\newcommand{\ccmanishort}[1]{{\sc #1-CCM}}
\newcommand{\dcmanishort}[1]{{\sc #1-DCM}}

\newcommand{\cmanishort}[1]{{\sc #1-CM}}
\newcommand{\dmanishort}[1]{{\sc #1-DM}}
\newcommand{\cwmanishort}[1]{{\sc #1-CWM}}
\newcommand{\dwmanishort}[1]{{\sc #1-DWM}}

\newcommand{\cmaniinst}{(C,V,S,c)}
\newcommand{\cwmaniinst}{(C,V,S,s,t,c)}

\newcommand{\Algo}[4]{
\medskip
\begin{center}
\begin{tabularx}{0.92\textwidth}{ll}
\hline
\multicolumn{2}{c}{\sc{#1}} \\
\hline
\em{Eingabe:}& #2\\
\em{Ausgabe:}& #3 \\
\em{Arbeitsweise:} & #4 \\
\hline
\end{tabularx}
\end{center}
\medskip}

\newcommand{\EP}[3]{
\medskip
\begin{center}
\begin{tabularx}{0.92\textwidth}{ll}
\hline\hline
\multicolumn{2}{c}{\sc{#1}} \\
\hline
\em{Gegeben:}& #2\\
\em{Frage:}& #3 \\
\hline\hline
\end{tabularx}
\end{center}
\medskip}

\newcommand{\ccwmaniprob}[1]{
\EP{\ccwmani{#1}}
{Eine #1-Wahl $(C,V\cup S)$, wobei $V$ die Menge der \\ 
& unmanipulierbaren W\"ahler ist  (diese Stimmen stehen fest) und $S$ die \\ 
& Menge der manipulierenden W\"ahler (diese Stimmen stehen nicht fest) \\ 
& und ein ausgezeichneter Kandidat $c$.}
{Ist es m\"oglich,die Stimmen in $S$ so zu w\"ahlen, dass $c$ ein \\ 
& #1-Gewinner in $(C,V\cup S)$ ist?}}
\newcommand{\xcovthreeprob}{
\EP{Exact Cover By Three Sets (X3C)}
{Eine Menge $B = \xSet{b}{3m}$, wobei $m\geq 1$, und eine Familie \\ 
& von Teilmengen $\coll = \xSet{S}{n}$ mit $\|S_{i}\| = 3 $ und $S_{i}\seq B$ f\"ur \\ 
& alle  $i$ mit $ 1\leq i 	\leq n.$}
{Gibt es eine Teilmenge $\coll'\seq \coll$ derart, dass jedes Element aus $B$ in  \\ 
& genau einer der Teilmengen aus $\coll'$ enthalten ist? %(Das hei"st, dass $\coll$  \\ 
%& f\"ur die Menge $B$ eine exakte \"Uberdeckung $\coll'$ enth\"alt.)
}}
\newcommand{\setcoverprob}{
\EP{Set Cover (SC)}
{Eine Menge $B = \xSet{b}{l}$, wobei $l\geq 1$, und eine Familie von \\ 
& Teilmengen $\coll = \xSet{S}{n}$ mit $S_i\seq B$ f\"ur alle $i$ mit\\ 
& $ 1\leq i \leq n$ und eine nat\"urliche Zahl $k$ mit $k\leq \|\coll\|.$}
{Gibt es eine Teilmenge $\coll'\seq \coll$ mit $\|\coll'\|\leq k$ derart, dass jedes \\ 
& Element aus $B$ in mindestens einer der Teilmengen aus $\coll'$ enthalten \\ 
& ist? %(Das hei"st, dass $\coll$ f\"ur die Menge $B$ eine \"Uberdeckung $\coll'$ enth\"alt.)
}}
\newcommand{\hittingsetprob}{
\EP{Hitting Set (HS)}
{Eine Menge $B = \xSet{b}{m}$, wobei $m\geq 1$, und eine Familie \\ 
&  von Teilmengen $\coll = \xSet{S}{n}$ mit $S_{i}\seq B$ f"ur alle  $i$ mit \\ 
& $ 1\leq i 	\leq n.$}
{Gibt es eine Teilmenge $B'\seq B$ derart, dass jedes Element aus $B'$ in \\ 
& mindestens einer der Teilmengen aus $\mathcal{S}$ enthalten ist? 
%(Das hei"st, dass $\mathcal{S}$ f"ur die Menge $B$ eine exakte \"Uberdeckung $\mathcal{S}'$ enth"alt.)
}}
\newcommand{\partitionprob}{
\EP{Partition}
{Eine Menge $A = \xSet{a}{n}$ und eine Funktion $s:A\rightarrow \mathbb{Z}$.}
{Gibt es eine Teilmenge $A'\seq A$ derart, dass gilt  $\sum\limits_{a\in A'}s(a) = \sum\limits_{a\not\in A'}s(a)$?}}
\newcommand{\subsetsumprob}{
\EP{Subset Sum (SS)}
{Eine Menge $A = \xSet{a}{n}$, eine Funktion $s:A\rightarrow \mathbb{Z}$ und eine  \\ 
& positive nat\"urliche Zahl $k$.}
{Gibt es eine Teilmenge $A'\seq A$ derart, dass gilt  $\sum\limits_{a\in A'}s(a) = k$?}}

%\pagestyle{empty}

\begin{document}
\vspace{0.5cm}

\begin{center}
{\Large Klausurvorschlag zur Vorlesung im WS 2010/2011} \\
{\LARGE \bf Algorithmische Eigenschaften von Wahlsystemen I} \\
\end{center}
\vspace{0.5cm}


%%%%%%%%%%%%%%%%%%%%%%%%%%%%%%%
%%%%%%%%   Aufgabe 1   %%%%%%%%
%%%%%%%%%%%%%%%%%%%%%%%%%%%%%%%
 \noindent{\bf Aufgabe 1 (Ankreuzaufgaben):} 
\newline 
\begin{tabular}{ll}
$\square$ $\square$&Die Gewinnerbestimmung von STV ist nicht in P.\\
$\square$ $\square$&F"ur ||$C$||=2 ist der PV Gewinner = Condorcet Gewinner.\\
$\square$ $\square$&Aus dem Konsistenz-Kriterium folgt: Gewinnt $c$ nicht in der gesamten Wahl, so\\&gibt es keine Aufteilung in zwei Unterwahlen in denen ausschlie"slich $c$ gewinnt.\\
$\square$ $\square$&Erlaube man beim Condorcet Wahlsystem auch Gleichst"ande bei der Gewinnerbe-\\&stimmung, so w"are es nicht mehr streng monoton.\\
$\square$ $\square$&Ein Veto-Gewinner ist nie ein Condorcet-Verlierer.\\
$\square$ $\square$&Young ist homogen.\\
$\square$ $\square$&Es gibt Wahlen in denen der Condorcet-Gewinner existiert, und er nicht\\&der Copeland-Gewinner ist.\\
$\square$ $\square$&Erf"ullt $k$-Approval f"ur alle $k$ mit $1 \leq k \leq ||C||-1$ ein Kriterium, so erf"ullt es auch Plurality.\\
$\square$ $\square$&Alle Scoring-Protokolle sind konsistent.\\
$\square$ $\square$&Es macht keinen Unterschied, ob man die Majorit"at von W"ahlern wie gewohnt als\\&$\lfloor {\frac{||V||}{2}} \rfloor +1$ oder als $\lceil{\frac{||V||}{2}} \rceil$ definiert.\\
\end{tabular}

\vspace{0.5cm}

\noindent\textbf{L\"osungsvorschlag:}
\newline
\begin{tabular}{ll}
$\square$ \frame{X}&Die Gewinnerbestimmung von STV ist nicht in P.\\
$\square$ \frame{X}&F"ur ||$C$||=2 ist der PV Gewinner = Condorcet Gewinner.\\
 \frame{X} $\square$&Aus dem Konsistenz-Kriterium folgt: Gewinnt $c$ nicht in der gesamten Wahl, so\\&gibt es keine Aufteilung in zwei Unterwahlen in denen ausschlie"slich $c$ gewinnt.\\
 \frame{X} $\square$&Erlaube man beim Condorcet Wahlsystem auch Gleichst"ande bei der Gewinnerbe-\\&stimmung, so w"are es nicht mehr streng monoton.\\
$\square$ \frame{X}&Ein Veto-Gewinner ist nie ein Condorcet-Verlierer.\\
$\square$ \frame{X}&Young ist homogen.\\
$\square$ \frame{X}&Es gibt Wahlen in denen der Condorcet-Gewinner existiert, und er nicht\\&der Copeland-Gewinner ist.\\
 \frame{X} $\square$&Erf"ullt $k$-Approval f"ur alle $k$ mit $1 \leq k \leq ||C||-1$ ein Kriterium, so erf"ullt es auch Plurality.\\
 \frame{X} $\square$&Alle Scoring-Protokolle sind konsistent.\\
$\square$  \frame{X}&Es macht keinen Unterschied, ob man die Majorit"at von W"ahlern wie gewohnt als\\&$\lfloor {\frac{||V||}{2}} \rfloor +1$ oder als $\lceil{\frac{||V||}{2}} \rceil$ definiert.\\
\end{tabular}
\newpage
%%%%%%%%%%%%%%%%%%%%%%%%%%%%%%%
%%%%%%%%   Aufgabe 2   %%%%%%%%
%%%%%%%%%%%%%%%%%%%%%%%%%%%%%%%
\noindent{\bf Aufgabe 2 (Scoring-Protokolle):}
\newline
Ein $Scoring$-$Protokoll$ ist f"ur eine Wahl mit $m$ Kandidaten definiert durch einen $Scoring$-$Vektor$ $\alpha=(\alpha_1,...,\alpha_m)$, der die folgende Bedingung erf"ullen muss: $$\alpha_1\geq ...\geq \alpha_m$$
Wird ein Kandidat in einer Stimme auf dem $k$-ten Platz einsortiert (f"ur $1\leq k \leq m$), so
erh"alt er aus dieser Stimme $\alpha_k$ Punkte. Der Kandidat mit der h"ochsten Punktzahl gewinnt.\\
Es sei ein Scoring-Protokoll $\alpha = (\alpha_1, \alpha_2,...,\alpha_m)$ gegeben.\\
\begin{enumerate}
\item Zeigen oder widerlegen Sie die folgenden beiden Aussagen:
\begin{itemize}
\item F"ur alle nat"urlichen Zahlen $k$ ist $\alpha_{+k'} := (\alpha_1 + k, \alpha_2 + k, . . . ,\alpha_{m-2}+k,\alpha_{m-1},\alpha_{m})$ ein
Scoring-Protokoll und es gilt f"ur alle Kandidaten $c \in C$ "uber jeder W"ahlermenge $V$:
Kandidat $c$ ist (eindeutiger) Gewinner in $(C, V)$ bez"uglich des Scoring-Protokolls $\alpha$ genau dann, wenn Kandidat $c$ (eindeutiger) Gewinner in $(C, V)$ bez"uglich des
Scoring-Protokolls $\alpha_{+k'}$ ist.
\end{itemize}
\item Begr"unden Sie warum es Scoring-Protokolle gibt oder nicht geben kann, f"ur welche $\alpha_{+k'}$ beide Eigenschaften erf"ullt. Sofern es m"oglich ist, geben Sie Beispiele aus der Vorlesung an.\\
\end{enumerate}

\vspace{0.5cm}

\noindent\textbf{L\"osungsvorschlag:}
\newline
Gegeben sei das Scoring-Protokoll $\alpha = (\alpha_1, \alpha_2,...,\alpha_m)$. Es gilt also $$\alpha_1\geq ...\geq \alpha_m$$ Sei $score_{\alpha}(c)$ die Punktzahl von $c$ in der Wahl ($C, V$) bez"uglich des Scoring-Protokolls $\alpha$.\\
\begin{enumerate}
\item Wir betrachten erstmal die zwei Teile von $\alpha_{+k'}$ separiert. Aus der obigen Eigenschaft und $k \geq 0$ folgt $\alpha_1+k\geq ...\geq \alpha_{m-2}+k$. Daraus folgt auch $\alpha_{m-2}+k \geq \alpha_{m-1}$ und $\alpha_{m-1} \geq \alpha_{m}$ gilt nach Voraussetzung. Damit gilt $\alpha_1 + k \geq \alpha_2 + k \geq . . . \geq \alpha_{m-2}+k \geq \alpha_{m-1} \geq \alpha_{m}$ und $\alpha_{+k'}$ ist ein Scoring-Protokoll.\\
Gegenbeispiel f"ur die Gewinnererhaltung:\\
Sei $\alpha$ $m$-1-Approval und die folgende Wahl gegeben:\\
\bc
\begin{tabular}{|lc|}
\hline
$C_1 =$&$ \{a,b,c,d,\}$ 	\\
 $V_1:$ 	& $a\,b\,c\,d$\\
		& $b\,d\,c\,a$\\
		& $a\,d\,c\,b$\\
\hline
\end{tabular}
\ec
Der Gewinner in $\alpha$ ist $c$ mit 3 Punkten. Sei nun $k$=1. Damit w"aren die Gewinner von $\alpha_{+k'}$ $\{a,b,d\}$ mit jeweils 4 Punkten.\\
\item Wahlsysteme bei denen Kandidaten, die auf den letzten zwei Positionen stehen, keine Punkte erhalten, und mindestens einen Punkt auf der ersten Position und die Punktzahl f"ur alle "ubrigen $\alpha_i$'s gleich ist, k"onnen ausgedr"uckt werden als\\$(\alpha_1, \alpha_2, . . . ,\alpha_{m-2},0,0)$. Jeder Kandidat kriegt bei $\alpha_{+k'}$ h"ochtens $(m-2)\cdot k$ Punkte mehr und trivialerweise kriegt keiner mehr als der Gewinner aus $\alpha$. Da wir aus der "Ubung wissen, dass bei einer gleichverteilten Zunahme von Punkten der Gewinner erhalten bleibt, gibt es solche Scoring-Protokolle wie z.B. $(m-2)$-Approval.
\end{enumerate}
\newpage
%%%%%%%%%%%%%%%%%%%%%%%%%%%%%%%
%%%%%%%%   Aufgabe 3   %%%%%%%%
%%%%%%%%%%%%%%%%%%%%%%%%%%%%%%%
\noindent{\bf Aufgabe 3 (Manipulation in Regular Cup):}
\newline
Gegeben sei die Regular Cup-Wahl 
$\wahlind{1}$: 
\bc
\begin{tabular}{|lc|}
\hline
$C_1 =$&$ \{a,b,c,d,e,f,g\}$ 	\\
 $V_1:$ 	& $e\,a\,d\,f\,c\,g\,b$		\\
		& $d\,e\,a\,b\,f\,c\,g$		\\
		& $g\,f\,b\,e\,c\,a\,d$		\\
 		& $a\,f\,e\,g\,c\,d\,b$		\\
 		& $b\,e\,d\,g\,a\,f\,c$		\\
\hline
\end{tabular}
\ec
Die Zuteilung der Bl\"atter erfolge alphabetisch von links nach rechts. In der ersten Runde tritt also $a$ gegen $b$ an und $c$ tritt gegen $d$ und $e$ tritt gegen $f$ an. Gleichst\"ande werden lexikographisch gebrochen: Es gewinnt der
lexikographisch kleinere Kandidat. Wir betrachten das eindeutige Gewinnermodell.\\
\begin{enumerate}
\item Bestimmen Sie den Gewinner der  Regular Cup-Wahl 
$\wahlind{1}$.
\item Entscheiden Sie, ob $(C_1,V_1,S,c)$ mit $\|S\|=2$ eine Ja-Instanz f\"ur \ccmani{Regular Cup} ist.
\item Entscheiden Sie ob ein Manipulator mit nur einer Stimme ausreichen w"urde um $c$ zum Gewinner zu machen. Begr"unden Sie ihre Entscheidung.
\end{enumerate}
Geben Sie bei einer Ja-Instanz die Stimmen der manipulierenden W\"ahler explizit an. Konstruieren Sie den entsprechenden Bin\"arb"aume.

\vspace{0.5cm}

\noindent\textbf{L\"osungsvorschlag:}
\begin{enumerate}
\item Der Bin"arbaum sieht folgend aus:\\
\begin{center}
		\unitlength1cm
		\thicklines
		\begin{picture}(7,4.5)
			\multiput(0.25,0.75)(1.0,0.0){6}{\circle{0.5}}
			\multiput(0.75,1.75)(2.0,0.0){4}{\circle{0.5}}
			\multiput(1.75,2.75)(4.0,0.0){2}{\circle{0.5}}
			\put(3.75,4.00){\circle{0.5}}
			%Blaetter
			\put(0.0,0.5){\makebox(0.5,0.5){a}}\put(0.0,0.0){\makebox(0.5,0.5){3}}
			\put(1.0,0.5){\makebox(0.5,0.5){b}}\put(1.0,0.0){\makebox(0.5,0.5){2}}
			\put(2.0,0.5){\makebox(0.5,0.5){c}}\put(2.0,0.0){\makebox(0.5,0.5){2}}
			\put(3.0,0.5){\makebox(0.5,0.5){d}}\put(3.0,0.0){\makebox(0.5,0.5){3}}
			\put(4.0,0.5){\makebox(0.5,0.5){e}}\put(4.0,0.0){\makebox(0.5,0.5){3}}
			\put(5.0,0.5){\makebox(0.5,0.5){f}}\put(5.0,0.0){\makebox(0.5,0.5){2}}
			\put(6.5,1.5){\makebox(0.5,0.5){g}}\put(6.5,1.0){\makebox(0.5,0.5){1}}
			%1. Knotenebene
			\put(0.5,1.5){\makebox(0.5,0.5){a}}\put(0.5,1.0){\makebox(0.5,0.5){3}}
			\put(2.5,1.5){\makebox(0.5,0.5){d}}\put(2.5,1.0){\makebox(0.5,0.5){2}}
			\put(1.5,2.5){\makebox(0.5,0.5){a}}\put(1.5,2.0){\makebox(0.5,0.5){1}}
			\put(4.5,1.5){\makebox(0.5,0.5){e}}\put(4.5,1.0){\makebox(0.5,0.5){4}}
			\put(5.5,2.5){\makebox(0.5,0.5){e}}\put(5.5,2.0){\makebox(0.5,0.5){4}}
			%Wurzel
			\put(3.5,3.75){\makebox(0.5,0.5){e}}\put(3.5,3.5){\makebox(0.5,0.5){}}
			%Kanten
			\put(4.25,1.00){\line(1,2){0.3}}
			\put(5.25,1.00){\line(-1,2){0.3}}
			\put(0.25,1.00){\line(1,2){0.3}}
			\put(1.25,1.00){\line(-1,2){0.3}}
			\put(2.25,1.00){\line(1,2){0.3}}
			\put(3.25,1.00){\line(-1,2){0.3}}
			\put(0.75,2.00){\line(4,3){0.8}}
			\put(2.75,2.00){\line(-4,3){0.8}}
			\put(4.75,2.00){\line(4,3){0.8}}
			\put(6.75,2.00){\line(-4,3){0.8}}
			\put(1.75,3.00){\line(2,1){1.8}}
			\put(5.75,3.00){\line(-2,1){1.8}}
		\end{picture}
	\end{center}
	Kandidat $e$ ist also Regular-Cup-Gewinner der Wahl $\wahlind{1}$. 
\item Angenommen, die beiden 
	W\"ahler in $S$ stimmen wie folgt ab: $c\,g\,f\,b\,a\,d\,e$ und $c\,g\,f\,b\,a\,d\,e$. Dann entsteht folgende Wahl 
	$(C_1,V_1\cup S)$:
\begin{center}
		\unitlength1cm
		\thicklines
		\begin{picture}(7,4.5)
			\multiput(0.25,0.75)(1.0,0.0){6}{\circle{0.5}}
			\multiput(0.75,1.75)(2.0,0.0){4}{\circle{0.5}}
			\multiput(1.75,2.75)(4.0,0.0){2}{\circle{0.5}}
			\put(3.75,4.00){\circle{0.5}}
			%Blaetter
			\put(0.0,0.5){\makebox(0.5,0.5){a}}\put(0.0,0.0){\makebox(0.5,0.5){3}}
			\put(1.0,0.5){\makebox(0.5,0.5){b}}\put(1.0,0.0){\makebox(0.5,0.5){4}}
			\put(2.0,0.5){\makebox(0.5,0.5){c}}\put(2.0,0.0){\makebox(0.5,0.5){4}}
			\put(3.0,0.5){\makebox(0.5,0.5){d}}\put(3.0,0.0){\makebox(0.5,0.5){3}}
			\put(4.0,0.5){\makebox(0.5,0.5){e}}\put(4.0,0.0){\makebox(0.5,0.5){3}}
			\put(5.0,0.5){\makebox(0.5,0.5){f}}\put(5.0,0.0){\makebox(0.5,0.5){4}}
			\put(6.5,1.5){\makebox(0.5,0.5){g}}\put(6.5,1.0){\makebox(0.5,0.5){4}}
			%1. Knotenebene
			\put(0.5,1.5){\makebox(0.5,0.5){b}}\put(0.5,1.0){\makebox(0.5,0.5){3}}
			\put(2.5,1.5){\makebox(0.5,0.5){c}}\put(2.5,1.0){\makebox(0.5,0.5){4}}
			\put(1.5,2.5){\makebox(0.5,0.5){c}}\put(1.5,2.0){\makebox(0.5,0.5){4}}
			\put(4.5,1.5){\makebox(0.5,0.5){f}}\put(4.5,1.0){\makebox(0.5,0.5){3}}
			\put(5.5,2.5){\makebox(0.5,0.5){g}}\put(5.5,2.0){\makebox(0.5,0.5){3}}
			%Wurzel
			\put(3.5,3.75){\makebox(0.5,0.5){c}}\put(3.5,3.5){\makebox(0.5,0.5){}}
			%Kanten
			\put(4.25,1.00){\line(1,2){0.3}}
			\put(5.25,1.00){\line(-1,2){0.3}}
			\put(0.25,1.00){\line(1,2){0.3}}
			\put(1.25,1.00){\line(-1,2){0.3}}
			\put(2.25,1.00){\line(1,2){0.3}}
			\put(3.25,1.00){\line(-1,2){0.3}}
			\put(0.75,2.00){\line(4,3){0.8}}
			\put(2.75,2.00){\line(-4,3){0.8}}
			\put(4.75,2.00){\line(4,3){0.8}}
			\put(6.75,2.00){\line(-4,3){0.8}}
			\put(1.75,3.00){\line(2,1){1.8}}
			\put(5.75,3.00){\line(-2,1){1.8}}
		\end{picture}
	\end{center}
	$c$ konnte zum eindeutigen Gewinner gemacht werden. Somit handelt es sich um eine Ja-
	Instanz.\\ 
\item Eine Stimme w"urde nicht ausreichen, denn $c$ braucht mindestens einen Punkt mehr als $a$ oder $b$ (da Gleichst\"ande lexikographisch gebrochen werden) um diese zu schlagen und damit sind mindestens zwei Stimmen des Manipulators notwendig.
\end{enumerate}
\vspace{0.5cm}
\newpage
%%%%%%%%%%%%%%%%%%%%%%%%%%%%%%%
%%%%%%%%   Aufgabe 4   %%%%%%%%
%%%%%%%%%%%%%%%%%%%%%%%%%%%%%%%
\noindent{\bf Aufgabe 4 (Das Wahlsystem Bucklin):} \newline
Gegeben sei eine Wahl ($C, V$). Wir definieren das folgende Wahlsystem:
\begin{itemize}
\item $Bucklin$: Der sogenannte Bucklin-Score eines Kandidaten der Stufe $i$ ist die Anzahl der W"ahler, die diesen Kandidaten innerhalb ihrer ersten $i$ Positionen stellen. Der Bucklin-Score von $c$ in einer Wahl $(C,V)$ ist das kleinste $i$ mit der Eigenschaft, dass der Kandidat bei mindestens einer Majorit"at von W"ahlern innerhalb der ersten $i$ Positionen steht. Alle Kandidaten mit dem kleinsten $i$ und dem h"ochsten Bucklin-Score der Stufe $i$ sind die Bucklin-Gewinner. 
\end{itemize}
Eigenschaften von Wahlsystemen: Ein Wahlsystem $\vs$ erf\"ullt das 
\renewcommand{\labelenumi}{(\arabic{enumi})}
\be
	\item \emph{Monotonie-Kriterium}, wenn f\"ur jede $\vs$-Wahl $\wahl$ gilt: 
	Ist Kandidat $c$ ein $\vs$-Gewinner in $\wahl$ und verbessern wir die Position von $c$ in 
	einigen Stimmen in $V$, wobei sonst keine Ver\"anderungen vorgenommen werden, so ist $c$ 
	ein $\vs$-Gewinner der ver\"anderten Wahl. 
 \item \emph{Mehrheitskriterium}, wenn in jeder $\vs$-Wahl immer derjenige Kandidat gewinnt, der von einer absoluten Mehrheit der W"ahler auf den ersten Platz gew"ahlt wird (sofern dieser existiert).
\item \emph{Konsistenz-Kriterium}, wenn f"ur jede $\vs$-Wahl ($C, V$) gilt: Wird die W"ahlermenge $V$ aufgeteilt in zwei disjunkte Teilmengen $V_1, V_2$ und ist Kandidat $c$ ein $\vs$-Gewinner
in beiden Unterwahlen $(C, V_1$) und ($C, V_2$), so ist $c$ auch in der Wahl ($C, V$ ) ein $\vs$-Gewinner.
\ee
Seien die folgenden Wahlen $E_2=(C_2,V_2)$ und $E_3=(C_3,V_3)$ gegeben:
\bc
\begin{tabular}{|lc|lc|}
\hline
\multicolumn{2}{|l|}{$C_2 =\{a,b,c,d,e\}$}& \multicolumn{2}{l|}{$C_3 = \{a,b,c,d,e\}$ }	\\
$V_2:$ 	& $a\,b\,c\,d\,e$			& $V_3:$ 	& $a\,b\,c\,e\,d$		\\
	& $a\,b\,c\,e\,d$			& 		& $a\,e\,c\,d\,b$		\\
	& $e\,d\,c\,b\,a$			& 		& $c\,a\,d\,e\,b$		\\
	& $e\,d\,c\,a\,b$			& 		& $c\,d\,a\,b\,e$		\\
	& 				& 		& $c\,b\,a\,e\,d$		\\\hline
\end{tabular}
\ec
\begin{enumerate}
\item[(a)] Bestimmen Sie die Bucklin Gewinner in den Wahlen $E_2$ und $E_3$.
\item[(b)] Zeigen Sie, oder widerlegen Sie, ob das Wahlsystem Bucklin die drei oben angegebenen Kriterien erf"ullt.
\end{enumerate}

\vspace{0.5cm}

\noindent\textbf{L\"osungsvorschlag:}
\begin{enumerate} 
\item[(a)] Der Stufe 3 Gewinner von  $E_2$ ist $c$. Der Stufe 1 Gewinner von  $E_3$ ist $c$.
\item[(b)] \textbf{Monotoniekriterium:} Jedes Vertauschen des Gewinners nach vorne, kann dessen Bucklin Stufe h"ochstens verringern. Gleichzeitig kann die Stufe der anderen an dem Tausch beteiligten Kandidaten entweder gleich bleiben oder sich erh"ohen.\\
 \textbf{Mehrheitskriterium:} Bucklin erf"ullt das Mehrheitskriterium, denn wenn es einen Kandidaten gibt, der in mehr als der H"alfte aller Stimmen auf dem ersten Platz platziert ist, so ist er eindeutiger Stufe 1 Gewinner.\\
\textbf{Konsistenz-Kriterium:} Bucklin erf"ullt nicht das Konsistenz-Kriterium, denn legen wir $E_2$ und $E_3$ zusammen, so ist $a$ Stufe 2 Bucklin Gewinner, obwohl $c$ beide Teilwahlen gewonnen hat. 
\end{enumerate}
\newpage
%%%%%%%%%%%%%%%%%%%%%%%%%%%%%%%
%%%%%%%%   Aufgabe 5   %%%%%%%%
%%%%%%%%%%%%%%%%%%%%%%%%%%%%%%%

\noindent{\bf Aufgabe 5 (DCWM f"ur STV und 3 Kandidaten):} \newline
In der Vorlesung haben Sie die
Reduktion von PARTITION auf das Manipulationsproblem DCWM f"ur das Wahlsystem
STV mit 3 Kandidaten kennengelernt. Gegeben seien folgende PARTITION-Instanzen: $(2,5,6,8,4,1)$ und $(5,5,3,7,8,10)$.
\begin{enumerate}
\item[(a)]{Konstruieren Sie gem"a"s der Reduktion die STV-Wahl ($C, V$) aus einer der oberen PARTITION-Instanzen. Bestimmen Sie die Stimmen und Gewichte der Manipulatoren in $S$ und berechnen Sie die STV-Punktwerte in den ersten Runden in den Wahlen ($C, V$) und ($C, V \cup S$), so dass eine Manipulation erfolgreich ist.}
\item[(b)]{Konstruieren Sie gem"a"s der Reduktion die STV-Wahl ($C', V'$) aus einer der oberen PARTITION-Instanzen. Bestimmen Sie die Gewichte der Manipulatoren in $S'$. Erl"autern Sie, weshalb hier keine erfolgreiche Manipulation m"oglich ist.}
\end{enumerate}

\vspace{0.5cm}

\noindent\textbf{L\"osungsvorschlag:}
\begin{enumerate}
\item[(a)]Damit eine Manipulation erfolgreich ist, m"ussen wir eine Ja-Instanz f"ur PARTITION verwenden. $(2,5,6,8,4,1)$ ist eine Ja-Instanz mit z.B. der Partition $(\{2,5,6\},\{8,4,1\})$.\\
$C=\{a,b,p\}, K=13;$
\begin{itemize}
\item es gibt 78 W"ahler der Form $a\ d\ b$ und\\
\item es gibt 78 W"ahler der Form $b\ d\ a$ und\\
\item es gibt 104 W"ahler der Form $d\ a\ b$.
\end{itemize}
$$score_{(C,V)}(a)=78 \ und \ score_{(C,V)}(b)=78 \ und\  score_{(C,V)}(c)=103$$
Die Stimmen der 6 Manipulatoren mit den Gewichten sehen folgend aus:\\
\bc
\begin{tabular}{|l|c|c|c|c|c|c|}
\hline
Manipulator $i$& 1&2&3&4&5&6\\
\hline
Gewicht&4&10&12&16&8&2\\
\hline
Pr"aferenz& $abd$&$abd$&$abd$&$bad$&$bad$&$bad$\\
\hline
\end{tabular}
\ec
$$score_{(C,V)}(a)=104 \ und \ score_{(C,V)}(b)=104 \ und\  score_{(C,V)}(c)=103$$
Also scheidet $c$ in der ersten Runde aus und die Manipulation ist erfolgreich.
\item[(b)] $(5,5,3,7,8,10)$ ist eine Nein-Instanz f"ur PARTITION.\\
 $C'=\{a,b,p\}, K=19;$
\begin{itemize}
\item es gibt 114 W"ahler der Form $a\ d\ b$ und\\
\item es gibt 114 W"ahler der Form $b\ d\ a$ und\\
\item es gibt 151 W"ahler der Form $d\ a\ b$.
\end{itemize}
$$score_{(C',V')}(a)=114 \ und \ score_{(C',V')}(b)=114 \ und\  score_{(C',V')}(c)=151$$
Die Stimmen der 6 Manipulatoren mit den Gewichten sehen z.B. folgend aus:\\
\bc
\begin{tabular}{|l|c|c|c|c|c|c|}
\hline
Manipulator $i$& 1&2&3&4&5&6\\
\hline
Gewicht&10&10&6&14&16&20\\
\hline
Pr"aferenz& $?$&$?$&$?$&$?$&$?$&$?$\\
\hline
\end{tabular}
\ec
Da es keine Ja-Instanz f"ur PARTITION ist, sind die Pr"aferenzen laut der Konstruktion nicht definiert.\\
Unabh"angig von der Verteilung der Pr"aferenzen gilt aber immer, dass der Kandidat $a$ oder der Kandidat $b$ weniger Stimmen bekommt als der Kandidat $d$ und damit in der ersten Runde ausscheiden wird. Laut der Konstruktion gilt, dass falls der Kandidat $d$ nicht in der ersten Runde ausscheidet, kann er nicht mehr am gewinnen gehindert werden und somit die Manipulation nicht erfolgreich sein.
\end{enumerate}
\newpage

%%%%%%%%%%%%%%%%%%%%%%%%%%%%%%%
%%%%%%%%   Aufgabe 6   %%%%%%%%
%%%%%%%%%%%%%%%%%%%%%%%%%%%%%%%
\noindent{\bf Bonusaufgabe 6 (Funktion $S_P$ f"ur Veto und $k$-Approval):}\newline
Es sei $\vs$ ein Wahlsystem, das f\"ur die Pr\"aferenz $P$ eines W\"ahlers jedem 
Kandidaten $c$ einen Punktwert $S_P(c)$ bez\"uglich $P$ zuweist.\\
Geben Sie f"ur die Wahlsysteme Veto und $k$-Approval die Funktion $S_P$ an.

\vspace{0.5cm}

\noindent\textbf{L\"osungsvorschlag:}
\newline
\textbf{$k$-Approval:} $S_{P}(a)=\lceil {\frac{||\{b \in C - {a}| a\  steht\  vor\  b\}||-n+k+1}{n}}\rceil$\\ 
\textbf{Veto:} $S_{P}(a)=\lceil {\frac{||\{b \in C - {a}| a\  steht\  vor\  b\}||}{n}}\rceil$ oder $S_{P}(a)=\{|\exists b \in C - {a}| a\  steht\  vor\  b|\}$
\end{document}

